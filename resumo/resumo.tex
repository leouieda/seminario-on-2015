\documentclass[12pt]{letter}

\usepackage{geometry}
\geometry{
	a4paper,
	total={210mm,297mm},
	left=20mm,
	right=20mm,
	top=25mm,
	bottom=20mm,
}

\usepackage[brazilian]{babel}
\usepackage[utf8]{inputenc}
\usepackage[T1]{fontenc}
\usepackage{fancyhdr}
\usepackage{graphicx}

\renewcommand{\headrulewidth}{0pt}
\fancyhead[L]{
	\includegraphics[width=5cm]{logo_on}
}

\begin{document}

\thispagestyle{fancy}

\begin{flushleft}

\textbf{Inversão gravimétrica do relevo da Moho em coordenadas esféricas}\\
\line(1,0){450}
\\\textbf{Estudante:} Leonardo Uieda\\
\line(1,0){450}
\\\textbf{Orientador:} Valéria C. F. Barbosa\\
\line(1,0){450}
\\\textbf{Nível:} DOUTORADO\\
\line(1,0){450}
\\\textbf{Período Previsto de Bolsa de Estudos:} 2011 - 2015\\
\line(1,0){450}
\\\textbf{Período a que se Refere o Relatório:} 2015\\
\line(1,0){450}

\end{flushleft}

\textbf{Resumo}


A descontinuidade de Mohorovičić (ou Moho), que separa a crosta do manto
terrestre, é estudada quase exclusivamente através de métodos geofísicos.
Os principais métodos utilizados para estimar a profundidade da Moho são a
sismologia (com fontes naturais ou artificiais) e a gravimetria.
A sismologia geralmente produz estimativas pontuais (através da análise da
dispersão de ondas de superfície ou funções do receptor) ou em áreas muito
restritas (através de levantamentos sísmicos ou inversões tomográficas).
Já a gravimetria é capaz de produzir estimativas regionais ou até mesmo
globais dependendo da cobertura de dados disponível.
O advento da gravimetria por satélite facilitou o acesso a dados de qualidade
com cobertura aproximadamente homogênea em áreas tradicionalmente pobres na
cobertura de dados terrestres.
No caso da América do Sul, a cobertura de dados sismológicos e de gravimetria
terrestre é heterogênea e geralmente concentrada em torno de centros urbanos e
da região costeira.
Neste trabalho, utilizaremos dados de gravimetria por satélite para mapear a
profundidade da Moho na América do Sul.
Para isso, desenvolvemos um método de inversão não-linear computacionalmente
eficiente e capaz de levar em consideração a curvatura da Terra.

Existem na literatura diversos métodos de inversão gravimétrica para estimar o
relevo de uma interface, como o embasamento de uma bacia sedimentar ou a Moho.
A maioria desses métodos discretizam o relevo em prismas retangulares retos
justapostos com densidade homogênea e conhecida.
O problema inverso é estimar a espessura desses prismas que melhor ajusta os
dados observados.
O uso de prismas retangulares implica em uma aproximação plana para a Terra,
que não é adequada para estudos em escala continental.
Uma melhor aproximação seria a de uma Terra esférica.
Nessa aproximação, o relevo da interface deve ser discretizado em tesseroides,
ou prismas esféricos.
Para ambas aproximações, o problema inverso resultante é custoso
computacionalmente por que requer a construção de grandes matrizes densas e a
solução de grandes sistemas lineares.
Exitem métodos que buscam aumentar a eficiência computacional dessa classe de
inversão gravimétrica.
Um desses métodos utiliza a Transformada Rápida de Fourier (FFT) para estimar o
relevo no domínio da frequência.
Outro método aproxima as matrizes envolvidas na inversão por matrizes
diagonais.
Essa aproximação elimina a necessidade efetuar operações matriciais e de
construir e resolver sistemas lineares, acelerando consideravelmente a
computação do resultado.
O método desenvolvido neste trabalho discretiza o relevo da Moho em tesseroides
e é baseado na aproximação por matrizes diagonais.
Além disso, nossa formulação utiliza regularização de suavidade para
estabilizar o problema inverso.
Nossa implementação acelera as operações matriciais e a solução de sistemas
lineares através do uso de matrizes esparsas.



\textbf{Palavras-Chave:}
Inversão. Gravimetria. Coordenadas esféricas. Tesseroide. Moho. América do Sul.

\end{document}
